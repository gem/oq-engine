By default, the scenario hazard calculator computes and stores
\glspl{acr:gmf} for each GMPE specified in the job configuration file. The
\glspl{acr:gmf} will be computed at each of the sites and for each of the
intensity measure types specified in the job configuration file.

Exporting the outputs from the \glspl{acr:gmf} in the xml format will produce
an xml file for each realisation containing the corresponding ground motion
fields. Listing~\ref{lst:output_gmf_scenario_xml} is an example of a \gls{acr:gmf}
collection NRML file containing one \gls{acr:gmf}:

\begin{listing}[htbp]
  \inputminted[firstline=1,firstnumber=1,fontsize=\footnotesize,frame=single,linenos,bgcolor=lightgray]{xml}{oqum/hazard/verbatim/output_gmf_scenario.xml}
  \caption{Example ground motion field collection output file for a scenario}
  \label{lst:output_gmf_scenario_xml}
\end{listing}

Exporting the outputs from the \glspl{acr:gmf} in the csv format results in
two csv files illustrated in the example files in
Table~\ref{output:gmf_scenario} and Table~\ref{output:sitemesh}. The sites csv
file provides the association between the site ids in the \glspl{acr:gmf} csv
file with their latitude and longitude coordinates.

\begin{table}[htbp]
\centering
\begin{tabular}{cccccc}

\hline
\rowcolor{lightgray}
\textbf{rlzi} & \textbf{sid} & \textbf{eid} & \textbf{gmv\_PGA} & \textbf{gmv\_SA(0.3)} & \textbf{gmv\_SA(1.0)} \\
\hline
0 & 0 & 0 & 6.211067E-02 & 1.194583E-01 & 7.427138E-02 \\
0 & 0 & 1 & 1.067956E-01 & 1.249612E-01 & 7.169625E-02 \\
0 & 0 & 2 & 8.719965E-02 & 1.024321E-01 & 1.676223E-01 \\
... & ... & ... & ... & ... & ... \\
1 & 6 & 97 & 2.317334E-01 & 9.786993E-02 & 2.835379E-01 \\
1 & 6 & 98 & 3.173075E-01 & 2.689785E-01 & 1.057350E-01 \\
1 & 6 & 99 & 5.139579E-01 & 3.395610E-01 & 2.365834E-01 \\
\hline

\end{tabular}
\caption{Example of a ground motion fields csv output file for a scenario}
\label{output:gmf_scenario}
\end{table}

\begin{table}[htbp]
\centering
\begin{tabular}{ccc}

\hline
\rowcolor{lightgray}
\textbf{site\_id} & \textbf{lon} & \textbf{lat} \\
\hline
0 & -122.57000 & 38.11300 \\
1 & -122.11400 & 38.11300 \\
2 & -122.00000 & 37.91000 \\
3 & -122.00000 & 38.00000 \\
4 & -122.00000 & 38.11300 \\
5 & -122.00000 & 38.22500 \\
6 & -121.88600 & 38.11300 \\
\hline

\end{tabular}
\caption{Example of a sites csv output file for a scenario}
\label{output:sitemesh}
\end{table}

