The \glsdesc{acr:oqe} output of a disaggregation analysis corresponds to the
combination of a hazard curve and a multidimensional matrix containing the
results of the disaggregation.

\begin{Verbatim}[frame=single, commandchars=\\\{\}, fontsize=\small]
user@ubuntu:~$ oq engine --lo <calc_id>
id | name
\textcolor{red}{3 | Disaggregation Outputs}
\textcolor{black}{4 | Realizations}
\end{Verbatim}
%\begin{Verbatim}[frame=single, commandchars=\\\{\}]
user@ubuntu:~$ oq engine --lo <calc_id>
id | output_type | name
19 | hazard_curve | hc-rlz-3
20 | hazard_curve | hc-rlz-3
21 | hazard_curve | hc-rlz-4
22 | hazard_curve | hc-rlz-4
23 | disagg_matrix | disagg(0.02)-rlz-3-SA(0.025)-POINT(10.1 40.1)
24 | disagg_matrix | disagg(0.1)-rlz-3-SA(0.025)-POINT(10.1 40.1)
25 | disagg_matrix | disagg(0.02)-rlz-3-PGA-POINT(10.1 40.1)
26 | disagg_matrix | disagg(0.1)-rlz-3-PGA-POINT(10.1 40.1)
27 | disagg_matrix | disagg(0.02)-rlz-4-SA(0.025)-POINT(10.1 40.1)
28 | disagg_matrix | disagg(0.1)-rlz-4-SA(0.025)-POINT(10.1 40.1)
29 | disagg_matrix | disagg(0.02)-rlz-4-PGA-POINT(10.1 40.1)
30 | disagg_matrix | disagg(0.1)-rlz-4-PGA-POINT(10.1 40.1)
\end{Verbatim}

Running \texttt{-{}-export-output} to export the disaggregation results will
produce individual files for each IMT, probability of exceedence and logic tree
realisation. In Listing~\ref{lst:output_hazard_map_xml}) we show an example of
the nrml file used to represent the different disaggregation matrices (highlighted
in red) produced by\gls{acr:oqe}:

\begin{listing}[htbp]
  \inputminted[firstline=1,firstnumber=1,fontsize=\footnotesize,frame=single,linenos,bgcolor=lightgray]{xml}{oqum/hazard/verbatim/output_disaggregation_matrix}
  \caption{Example of different disaggregation matrices produced by oq-engine}
  \label{lst:output_uhs}
\end{listing}
