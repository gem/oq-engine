The Event Based PSHA calculator computes and stores stochastic event sets and
the corresponding ground motion fields. This calculator can also produce
hazard curves and hazard maps, similar to the 
Classical PSHA calculator. The inset below shows an example of the list of
results provided by the \gls{acr:oqe} at the end of an event-based PSHA
calculation:

\begin{Verbatim}[frame=single, commandchars=\\\{\}, fontsize=\small]
user@ubuntu:~$ oq engine --lo <calc_id>
id | output_type | name
\textcolor{red}{10 | datastore  | gmfs}
11 | datastore  | hcurves
12 | datastore  | realizations
\textcolor{green}{13 | datastore  | sescollection}
\end{Verbatim}

Exporting the outputs from the ruptures will produce a CSV file containing
similar to the one is shown below in Listing~\ref{lst:output_ruptures}.

\begin{Verbatim}[frame=single, commandchars=\\\{\}, fontsize=\small]
<?xml version="1.0" encoding="UTF-8"?>
<nrml xmlns:gml="http://www.opengis.net/gml"
	  xmlns="http://openquake.org/xmlns/nrml/0.5">
  <ruptureCollection sourceModelTreePath="b1"
    \textcolor{red}{investigationTime="50.0">}
    <ruptureGroup id="0" tectonicRegion="Active Shallow Crust">
\textcolor{blue}{    <singlePlaneRupture id="13681" multiplicity="1">}
\textcolor{red}{       <stochasticEventSets>}
\textcolor{red}{           <SES id="7">}
\textcolor{red}{               34359738368}
\textcolor{red}{           </SES>}
\textcolor{red}{       </stochasticEventSets>}
\textcolor{blue}{       <magnitude>}
\textcolor{blue}{           5.05}
\textcolor{blue}{       </magnitude>}
\textcolor{blue}{       <strike>}
\textcolor{blue}{           0.0}
\textcolor{blue}{       </strike>}
\textcolor{blue}{       <dip>}
\textcolor{blue}{           90.0}
\textcolor{blue}{       </dip>}
\textcolor{blue}{       <rake>}
\textcolor{blue}{           0.0}
\textcolor{blue}{       </rake>}
\textcolor{blue}{       <hypocenter depth="5" lat="-0.20469" lon="-0.09530"/>}
\textcolor{blue}{       <planarSurface>}
\textcolor{blue}{           <topLeft depth="3.1741" lat="-0.221115" lon="-0.095302"/>}
\textcolor{blue}{           <topRight depth="3.1741" lat="-0.18827" lon="-0.095302"/>}
\textcolor{blue}{           <bottomLeft depth="6.82587" lat="-0.221115" lon="-0.095302"/>}
\textcolor{blue}{           <bottomRight depth="6.82587" lat="-0.188274" lon="-0.095302"/>}
\textcolor{blue}{       </planarSurface>}
\textcolor{blue}{    </singlePlaneRupture>}
    </ruptureGroup>
  </ruptureCollection>
</nrml>
\end{Verbatim}

\begin{listing}[htbp]
  \inputminted[firstline=1,firstnumber=1,fontsize=\footnotesize,frame=single,linenos,bgcolor=lightgray]{xml}{oqum/hazard/verbatim/output_ses.csv}
  \caption{Example of CSV file containing a collection of ruptures}
  \label{lst:output_ruptures}
\end{listing}

The outputs from the \glspl{acr:gmf} can be exported in the csv
format. Exporting the outputs from the \glspl{acr:gmf} in the 
csv format results in two csv files illustrated in the example files in
Table~\ref{output:gmf_event_based} and Table~\ref{output:sitemesh}. The sites csv
file provides the association between the site ids in the \glspl{acr:gmf} csv
file with their latitude and longitude coordinates.

\begin{table}[htbp]
\centering
\begin{tabular}{cclccc}

\hline
\rowcolor{lightgray}
\textbf{rlz\_id} & \textbf{site\_id} & \textbf{event\_id} & \textbf{gmv\_PGA} & \textbf{gmv\_SA(0.3)} & \textbf{gmv\_SA(1.0)} \\
\hline
0 & 0 & 48 & 0.0089 & 0.0686 & 0.0065 \\
0 & 0 & 54 & 0.0219 & 0.0325 & 0.0164 \\
… & … & … & … & … & … \\
0 & 6 & 75 & 0.0246 & 0.0244 & 0.0036 \\
1 & 0 & 76 & 0.0189 & 0.0327 & 0.0094 \\
1 & 0 & 77 & 0.0286 & 0.0683 & 0.0471 \\
… & … & … & … & … & … \\
… & … & … & … & … & … \\
7 & 6 & 1754 & 0.3182 & 1.2973 & 0.6127 \\
7 & 6 & 1755 & 0.2219 & 0.6200 & 0.5069 \\
\hline

\end{tabular}
\caption{Example of a ground motion fields csv output file for an event based hazard calculation}
\label{output:gmf_event_based}
\end{table}


The \texttt{Events} output produces a csv file with fields \texttt{event\_id},
\texttt{rup\_id} and \texttt{rlz\_id}. The \texttt{event\_id} is a 32 bit
integer that identifies uniquely the event; the \texttt{rup\_id}
is a 32 bit integer that identifies uniquely the rupture; the \texttt{rlz\_id}
is a 16 bit integer that identifies uniquely the
realization. The association between the \texttt{event\_id} and
the  \texttt{rup\_id} is stored inside the \texttt{Events} output.

The \texttt{Realizations} output produces a csv file listing the source model
and the combination of ground shaking intensity models for each path sampled
from the logic tree. An example of such a file is shown below in
Table~\ref{output:realizations}.

\begin{table}[htbp]
\centering
\begin{tabular}{cclll}

\hline
\rowcolor{lightgray}
\textbf{ordinal} & \textbf{branch\_path} & \textbf{weight} \\
\hline
0 & b1$\sim$b11\_b21 & 0.1125 \\
1 & b1$\sim$b11\_b22 & 0.075 \\
2 & b1$\sim$b12\_b21 & 0.0375 \\
3 & b1$\sim$b12\_b22 & 0.025 \\
4 & b2$\sim$b11\_b21 & 0.3375 \\
5 & b2$\sim$b11\_b22 & 0.225 \\
6 & b2$\sim$b12\_b21 & 0.1125 \\
7 & b2$\sim$b12\_b22 & 0.075 \\
\hline

\end{tabular}
\caption{Example of a realizations file}
\label{output:realizations}
\end{table}

