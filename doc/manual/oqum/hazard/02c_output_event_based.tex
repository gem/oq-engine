The Event Based PSHA calculator computes and stores stochastic event sets and
the corresponding ground motion fields.

This calculator can also produce hazard curves and hazard maps exactly in the
same way as done using the Classical PSHA calculator.

The inset below shows an example of the list of results provided by the
\gls{acr:oqe} at the end of an event-based PSHA calculation:

\begin{Verbatim}[frame=single, commandchars=\\\{\}, fontsize=\small]
user@ubuntu:~$ oq engine --lo <calc_id>
id | output_type | name
\textcolor{red}{10 | datastore  | gmfs}
11 | datastore  | hcurves
12 | datastore  | realizations
\textcolor{green}{13 | datastore  | sescollection}
\end{Verbatim}

This list in the inset above contains a set of stochastic events (in green)
and their corresponding sets of ground motion fields (in red).

%The whole group of stochastic event set and ground motion fields can be
%exported immediately using the results with \texttt{id} 35 and 25, %respectively.
Exporting the outputs from the stochastic event set will produce, for each realisation, a nrml file containing a collection of stochastic
event sets. Below is an example of the nrml file for a stochastic event set containing only 2 ruptures:

\begin{Verbatim}[frame=single, commandchars=\\\{\}, fontsize=\small]
<?xml version="1.0" encoding="UTF-8"?>
<nrml xmlns:gml="http://www.opengis.net/gml"
	  xmlns="http://openquake.org/xmlns/nrml/0.5">
  <ruptureCollection sourceModelTreePath="b1"
    \textcolor{red}{investigationTime="50.0">}
    <ruptureGroup id="0" tectonicRegion="Active Shallow Crust">
\textcolor{blue}{    <singlePlaneRupture id="13681" multiplicity="1">}
\textcolor{red}{       <stochasticEventSets>}
\textcolor{red}{           <SES id="7">}
\textcolor{red}{               34359738368}
\textcolor{red}{           </SES>}
\textcolor{red}{       </stochasticEventSets>}
\textcolor{blue}{       <magnitude>}
\textcolor{blue}{           5.05}
\textcolor{blue}{       </magnitude>}
\textcolor{blue}{       <strike>}
\textcolor{blue}{           0.0}
\textcolor{blue}{       </strike>}
\textcolor{blue}{       <dip>}
\textcolor{blue}{           90.0}
\textcolor{blue}{       </dip>}
\textcolor{blue}{       <rake>}
\textcolor{blue}{           0.0}
\textcolor{blue}{       </rake>}
\textcolor{blue}{       <hypocenter depth="5" lat="-0.20469" lon="-0.09530"/>}
\textcolor{blue}{       <planarSurface>}
\textcolor{blue}{           <topLeft depth="3.1741" lat="-0.221115" lon="-0.095302"/>}
\textcolor{blue}{           <topRight depth="3.1741" lat="-0.18827" lon="-0.095302"/>}
\textcolor{blue}{           <bottomLeft depth="6.82587" lat="-0.221115" lon="-0.095302"/>}
\textcolor{blue}{           <bottomRight depth="6.82587" lat="-0.188274" lon="-0.095302"/>}
\textcolor{blue}{       </planarSurface>}
\textcolor{blue}{    </singlePlaneRupture>}
    </ruptureGroup>
  </ruptureCollection>
</nrml>
\end{Verbatim}


The text in red shows the part which describes the id of the generated
stochastic event set and the investigation time covered.

The text in green emphasises the portion of the text used to describe a
rupture. The information provided describes entirely the geometry of the rupture as well as its rupturing properties (e.g. rake, magnitude). The rupture ID is a string that represents each rupture uniquely (including the case where the same rupture is sampled multiple times). The exact format may be subject to change from time-to-time; however, the ID will usually contain information regarding the specific logic tree branch sample (realisation), the stochastic event set counter, the ID of the source from which the rupture was samples and a unique number indicating the rupture identifier within the earthquake rupture forecast of the specific source.

Exporting the outputs from the gmfs will produce an xml file for each realisation containing the corresponding ground motion fields. Below is an example of a gmf collection nrml file containing one ground motion field:

\begin{Verbatim}[frame=single, commandchars=\\\{\}, fontsize=\small]
<?xml version="1.0" encoding="UTF-8"?>
<nrml xmlns:gml="http://www.opengis.net/gml"
      xmlns="http://openquake.org/xmlns/nrml/0.5">
  <gmfCollection sourceModelTreePath="b1" gsimTreePath="b1">
    <gmfSet investigationTime="50.0" stochasticEventSetId="12">
      <gmf IMT="PGA" ruptureId="col=00~ses=0001~src=10138~rup=0049-00">
        <node gmv="0.0105891230432" lon="11.1240023202"
            lat="43.5107462335"/>
        <node gmv="0.00905803920023" lon="11.1241875202"
            lat="43.6006783941"/>
        <node gmv="0.00637664420977" lon="11.1243735810"
            lat="43.6906105547"/>
        <node gmv="0.00476533134789" lon="11.1245605075"
            lat="43.7805427153"/>
        <node gmv="0.00452594698469" lon="11.1247483046"
            lat="43.8704748759"/>
        ...
        <node gmv="0.00017301076646" lon="11.3782630185"
            lat="44.5129482397"/>
      </gmf>
    </gmfSet>
  </gmfCollection>
</nrml>
\end{Verbatim}
