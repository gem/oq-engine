The classical PSHA-based risk calculator convolves through numerical
integration, the probabilistic vulnerability functions for an \gls{asset} with
the seismic hazard curve at the location of the asset, to give the loss
distribution for the asset within a specified time period. The calculator
requires the definition of an \gls{exposuremodel}, a \gls{vulnerabilitymodel} for
each loss type of interest with \glspl{vulnerabilityfunction} for each taxonomy
represented in the \gls{exposuremodel}, and hazard curves calculated in the
region of interest. Loss curves and loss maps can currently be calculated for
five different loss types using this calculator: structural losses,
nonstructural losses, contents losses, downtime losses, and occupant
fatalities. The main results of this calculator are loss exceedance curves for
each asset, which describe the probability of exceedance of different loss
levels over the specified time period, and loss maps for the region, which
describe the loss values that have a given probability of exceedance over the
specified time

Unlike the probabilistic event-based risk calculator, an aggregate loss curve
(considering all assets in the \gls{exposuremodel}) can not be extracted using
this calculator, as the correlation of the ground motion residuals and
vulnerability uncertainty is not taken into consideration in this calculator.

The hazard curves required for this calculator can be calculated by the
\glsdesc{acr:oqe} for all asset locations in the \gls{exposuremodel} using the
classical PSHA approach \citep{cornell1968, mcguire1976}. The use of logic-
trees allows for the consideration of model uncertainty in the choice of a
ground motion prediction equation for the different tectonic region types in
the region. Unlike what was described in the previous calculator, a total loss
curve (considering all assets in the \gls{exposuremodel}) can not be extracted
using this calculator, as the correlation of the ground motion residuals and
vulnerability uncertainty is not taken into consideration.

The required input files required for running a classical probabilistic risk
calculation and the resulting output files are depicted in
Figure~\ref{fig:io-structure-classical-risk}.

\begin{figure}[ht]
\centering
\includegraphics[width=9cm,height=7cm]{figures/risk/io-structure-classical-risk.pdf}
\caption{Classical PSHA-based Risk Calculator input/output structure.}
\label{fig:io-structure-classical-risk}
\end{figure}